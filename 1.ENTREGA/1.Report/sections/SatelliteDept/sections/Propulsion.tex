\subsection{Propulsion Systems}

\paragraph{}
Thruster is a main part of the structure because it is needed to allow the satellite to realise different maneuvers how incorporate it adequatly to the orbit after the deployment of the rocket, can obtain the optimal orientation or to mantain the satellite in the orbital and avoid its fallen. 

\paragraph{}
The main parameters that must consider are thrust, total specific impulse, power required, weight of the  propulsion subsystem and its volume.

\paragraph{}
At the moment, the most used and more modern thrusters for satellites are: ionic, pulsed plasma, electrothermal and green monopropellant thrusters. An important aspect to consider is that the goal is to reduce the mass required although this will cause minor accelerations than conventional engines but it will be suitable for small satellites.

\paragraph{}
BGT-X5 has been chosen how the CubeSat thruster. With the high thrust and delta V that BGT-X5 provides, the CubeSat will be able to carry out the necessary actions to keep the satellite in orbit, to relocate the satellite or to change its orbit.

\paragraph{}The option chosen is presented in the table below \ref{propulsionfinal}.

\begin{longtable}{| l | r | r | }
\hline
\rowcolor[gray]{0.80}	\textbf{System} &  \textbf{Brand and model}     & \textbf{Price per unit (\euro)}   \\
\hline
\endfirsthead

	   ~Propulsion & Busek BGT-X5 & 50000) \\
	\hline

\caption{Option chosen for the propulsion system}
\label{propulsionfinal}
\end{longtable}
