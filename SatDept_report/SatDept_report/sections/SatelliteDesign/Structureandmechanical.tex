\subsection{Structure and mechanics}

\paragraph{}The design and operation of a CubeSat is a complex process that must be completed keeping in mind the different subsystems it has as well as their role within the satellite. Additionally, since these systems will operate in space, they have to be prepared to withstand the extreme conditions, and be certified to work properly under them. 

\paragraph{}The satellite used by Astrea must have high compatibility between all the systems to avoid potential problems during operation and has to be tested (either all the systems together or one by one) and ensure their correct functioning.


\begin{figure}[h]
\includegraphics[scale=0.6]{./sections/SatelliteDesign/images/CubeSatDesign}
\centering
\caption{Dimensions of a 1U CubeSat \cite{cubesatdimensions}}
\end{figure}

\subsubsection{Structure}

\paragraph{}The mission of the structure is to sustain and protect all the electronic devices carried by the satellite in order to fulfill the mission requirements. In order to ensure that all the electronic and mechanic systems can be mounted upon the structure, a high compatibility between these systems is required.

The structure used will be provided by Innovative Solutions In Space (ISIS). Among its features it is worth mentioning that it can withstand the high range of temperature it will face in the space (-40ºC to 80ºC) and it is highly compatible with almost every physical system that will be used. Additionally, it is a low mass structure (304.3g) and it can support different configurations within it.

\paragraph{}Since the configuration within the CubeSat will not be as common as other configuration options within of current operational CubeSats because the mission of the project is to relay fast and reliable communication with the ground station and the other satellites, it is a really important point that the structure is highly flexible regarding the arrangement of the subsystems that it will carry.

\subsubsection{Thermal protection}
\paragraph{}The thermal protection system consists of various insulating materials that aim to protect the CubeSat from potential thermal shocks. The satellite must remain within an optimal range of temperature, despite of the variation of the external temperature, in order to work properly. Operating in space, the CubeSat is vulnerable to suffer extreme temperatures, both below zero and above zero, and thermal protection must guarantee that all subsystems are protected. Furthermore, the thermal protection system should also dissipate the heat produced by the other systems.

\paragraph{} Currently, the most used element as thermal protection in the aerospace industry is the multilayer insulation (MLI), a set of multiple thin insulation layers. The MLI fulfills all the requirements that were previously stated and its main objective is to reduce the heat generated by radiation since the heat generated by convection or conduction does not have such a high impact on the on-board systems.

\paragraph{} 
After a market study, Dunmore Aerospace company has been chosen to provide us its MLI product. Specially, the product is the Dunmore Aerospace Satkit and it is made for small satellites for low earth orbit.

\subsubsection{Study of the commercial available options}
\paragraph{}A broad marked study is needed since all the options have to be considered. For this reason, and with the aim to show all the information and features of each system that has been considered in this section, the table \ref{structureoptions} is presented below.


\begin{longtable}{| l | c | c | }
\hline
\rowcolor[gray]{0.80}	\textbf{Brand and model} &  \textbf{Features}     & \textbf{Total price (\euro)}   \\
\hline
\endfirsthead

\rowcolor[gray]{0.85} \textbf{Structure} &  &  \\
	   ~ISIS 3U structure & \makecell{Low mass (304.3g) \\ Highly compatible \\ High temperature range} & 3900 \\
	   \hline
	   ~Gomspace GOMX-Platform & \makecell{High mass (1500g) \\ Comes fully equipped (basic systems) \\ High temperature range} & \textbf{TO REQUEST!} \\
	   \hline
\rowcolor[gray]{0.85} \textbf{Thermal protection} &  &  \\
	   ~Dunmore Aerospace Satkit & \makecell{Lightweight \\ Durability \\ Made for small satellites}& \textbf{TO REQUEST!} \\
	   \hline
	   ~Dupont Kapton Aircraft Thermal & \makecell{Lightweight \\ Durability \\ Non-flammable} & \textbf{TO REQUEST!} \\
	\hline

\caption{Options studied}
\label{structureoptions}
\end{longtable}

\paragraph{}Finally, the options chosen are presented in the table \ref{structurefinal}.

\begin{longtable}{| l | r | r | }
\hline
\rowcolor[gray]{0.80}	\textbf{System} &  \textbf{Brand and model}     & \textbf{Price per unit (\euro)}   \\
\hline
\endfirsthead

	   ~3U Structure & ISIS & 3900) \\
	   \hline
	   ~Thermal Protection & Dunmore Satkit & TO REQUEST \\
	\hline

\caption{Options chosen}
\label{structurefinal}
\end{longtable}
