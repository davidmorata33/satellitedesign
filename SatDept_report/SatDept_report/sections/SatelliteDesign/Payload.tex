\subsection{Payload}

\subsubsection{Antennas}
\paragraph{}The antennas are essential in this mission, since their role is to transmit and receive the data from other satellites as well as the ground stations. In order to provide fast and reliable communication, several options have been studied and information about their main parameters is presented below.\\
It has to be kept in mind that the mass of the antennas should be as low as possible given that there are already a lot of subsystems in the CubeSat and the mass limitation is about 4kg.
Additionally, the power consumption has to be kept as low as possible given the limitations regarding the power supply that the CubeSat has and the antennas must be certified to work under space conditions (high temperature range and radiation protection shield).

\subsubsubsection{Basic parameters}
\paragraph{}The frequency range is one of the most important parameters, since it is related to an effective satellite-satellite and satellite-ground station communication. The frequency range should be between 1GHz and 10GHz, which is a very demanding condition given that the CubeSat has a limited space and power supply. \\
For an effective communication, the signal has to be able to trespass the atmosphere without a high number of losses and interference. The high frequency range allows the signal to go through this barrier and reach the ground stations.

\paragraph{}The bandwidth is the frequency range in which the highest power of the signal is found. It is really important to have a high bandwidth to have a great performance and avoid extremely high signal losses.

\paragraph{}The gain of an antenna is the ratio between the power density radiated in one direction and the power density that would radiate an isotropic antenna. The best option is to have a high gain. 

\paragraph{}The polarization of an antenna is the orientation of the electromagnetic waves when they are leaving it. There are three types of polarization: linear, circular and elliptical. For a high performance, the receiver antenna and the transmitter antenna should have the same polarization. It has been derived that the best option for the project is an antenna with circular polarization; these types of antennas are able to keep the signal constant regardless of the appearance of different adverse situations such as the relative movement of the satellites with respect to the ground station. 

\subsubsubsection{Patch antenna}

\paragraph{}\textbf{FALTA DESCRIPCIÓ}
\textbf{Nando, ¿Esto se puede arreglar un poco?}

\begin{longtable}{| l | r |}

\hline
\rowcolor[gray]{0.60}	\textbf{Patch antenna} \\
\hline

\hline
\rowcolor[gray]{0.75}	\textbf{Features} &  \textbf{Value}   \\
\hline

\cellcolor[gray]{0.85} \textbf{Bands} & L,S,C,X  \\
\cellcolor[gray]{0.85} \textbf{Frequency range} & 1-12 GHz  \\
\cellcolor[gray]{0.85} \textbf{Bandwidth} & 20 MHz \\
\cellcolor[gray]{0.85} \textbf{Gain} & 6 dBi  \\
\cellcolor[gray]{0.85} \textbf{Polarization} & Circular \\
\cellcolor[gray]{0.85} \textbf{Maximum power consumption} & 10 W \\
\cellcolor[gray]{0.85} \textbf{Impedance} & 50 Ohms \\
\cellcolor[gray]{0.85} \textbf{Operational temperature range} & -65ºC to +100ºC \\
\cellcolor[gray]{0.85} \textbf{Mass} & <250 grams \\
\hline
\caption{Main features of the patch antenna}
\label{patchantenna}
\end{longtable}

\subsubsubsection{Turnstile antenna}

\paragraph{}\textbf{FALTA DESCRIPCIÓ}
\textbf{Nando, ¿esto se puede arreglar un poco?}

\begin{longtable}{| l | r |}

\hline
\rowcolor[gray]{0.60}	\textbf{Turnstile antenna} \\
\hline

\hline
\rowcolor[gray]{0.75}	\textbf{Features} &  \textbf{Value}   \\
\hline

\cellcolor[gray]{0.85} \textbf{Frequency range} & 400-480 MHz  \\
\cellcolor[gray]{0.85} \textbf{Bandwidth} & 5 MHz \\
\cellcolor[gray]{0.85} \textbf{Gain} & 1.5 dBi \\
\cellcolor[gray]{0.85} \textbf{Polarization} & Circular \\
\cellcolor[gray]{0.85} \textbf{Maximum power consumption} & 10 W \\
\cellcolor[gray]{0.85} \textbf{Impedance} & 50 Ohms \\
\cellcolor[gray]{0.85} \textbf{Operational temperature range} & -40ºC to +85ºC \\
\cellcolor[gray]{0.85} \textbf{Mass} & 30 grams \\
\hline

\caption{Main features of the turnstile antenna}
\label{turnstileantenna}

\end{longtable}


\subsubsection{Data Handling Systems}
\textbf{FALTA!}

\paragraph{}The communication system allows the satellite to receive and transmit the data. It consists of a group of transponders, that are the combination of a transmitter and a receiver and whose functions are receiving, separating, amplifying, processing, reamplifying and retransmitting the signals.
\paragraph{} The telemetry subsytem analyses the information of the ground station and other sensors of the satellite in 	order to monitor the onboard conditions. With this system, the CubeSat is able to transmit the status of the onboad systems to the ground station.
\paragraph{}The command and control subsystem allows the ground station to control the satellite.

\subsubsection{Study of the commercial available options and options chosen}
\paragraph{}A broad marked study is needed since all the options have to be considered. For this reason, and with the aim to show all the information and features of each system that has been considered in this section, the table \ref{payloadoptions} is presented below.

\begin{longtable}{| l | c | c | }
\hline
\rowcolor[gray]{0.80}	\textbf{Brand and model} &  \textbf{Features}     & \textbf{Total price (\euro)}   \\
\hline
\endfirsthead

\rowcolor[gray]{0.85} \textbf{Transceiver} &  &  \\
	   ~	\textbf{POSAR!!!} & \makecell{\textbf{POSAR!!}\\ \textbf{POSAR!!}} & 17000 \\
	   \hline
	   ~	\textbf{POSAR!!!} & \makecell{\textbf{POSAR!!}\\ \textbf{POSAR!!}} & 17000 \\
	   \hline
	\hline
	
\caption{Options studied}
\label{payloadoptions}
\end{longtable}

Finally, with the aim to clarify all the information of this section, the chosen systems and components are presented in the table \ref{payloadchosen}.

\begin{longtable}{| l | r | r | r |}
	\hline
	\rowcolor[gray]{0.80}	\textbf{System} &  \textbf{Brand and model}     & \textbf{Price per unit (\euro)} & \textbf{N. of units}  \\
	\hline
	\endfirsthead
	
	~Patch antenna & ADC, Microstrip & TO REQUEST! & 4 \\
	~Turnstile antenna 1 & Gomspace, ANT430 & TO REQUEST & 2 \\
	~TRANSCEIVER & EMPTY & TO REQUEST & 1 \\
	~DHS & EMPTY & 20000 & 1 \\
	\hline
	
\caption{Options chosen}
\label{payloadchosen}
\end{longtable}

